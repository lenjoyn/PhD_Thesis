%% Title
%\titlepage[of Peterhouse]{./Peterhouse.png}{
%  A dissertation submitted to the University of Cambridge\\ for the degree of Doctor of Philosophy}
%% Abstract
\begin{center}%
	%\vspace*{\frontmattertitleskip}%
	\begin{doublespace}%
		{\Huge\textbf{\thetitle}}\\%
	\end{doublespace}%
	\vspace*{3cm}%
	{\Large{\theauthor} \\ of Peterhouse }\\%
	\vspace*{3cm}%
	\includegraphics[width=5cm]{Peterhouse.png} \\
	\vspace*{2cm}
	{A thesis submitted to the University of Cambridge\\ for the degree of Doctor of Philosophy \\ Aug 2019}
\end{center}%
\begin{abstract}%[\smaller \thetitle\\ \vspace*{1cm} \smaller {\theauthor}]
  %\thispagestyle{empty}
The Standard Model has been a successful theory in describing the behaviour of fundamental particles, but there are still problems remaining unsolved. New theoretical models are therefore proposed to answer those questions with either new interactions or new particles. This thesis is presenting the searches for new physics with diboson signatures in these two ways from LHC $\sqrt{s}=13~TeV$ collisions with the ATLAS detector with the data collected in 2015 and 2016 corresponding to an integrated luminosity of $36.1~fb^{-1}$. The searching strategy was performed with the Monte Carlo simulation for the SM background modelling and a data-driven method for the multijet background estimation. The final result was interpreted by a comparison between background modelling and data by a CLs method. For the resonance search, no new particle was discovered, and mass limits are therefore set on the new particles from models taken as the interpretation benchmarks. For the study on new interactions in the signatures of vector boson scattering, the first measurement of this process with the semileptonic decay was given with a significance of $2.7\sigma$ in reasonable agreement with the SM prediction
\\
\\Both the LHC and ATLAS detector are now going through the upgrades for operations in 2021 with the $\sqrt{s}=14~TeV$ collisions. The ATLAS hardware calorimeter trigger is part of the upgrade project for the implementation of three new object processors: eFex, jFex, and gFex. This thesis will also present the construction of simulation system along with the expected performance of proposed object reconstruction algorithms for this new infrastructure.

\end{abstract}
%% Declaration
\begin{declaration}
  This thesis is the result of my own work, except where explicit
reference is made to the work of others, and has not been submitted
for another qualification to this or any other university. This
thesis does not exceed the word limit for the respective Degree
Committee.
  \vspace*{1cm}
  \begin{flushright}
    Chiao-Ying Lin
  \end{flushright}
\end{declaration}


%% Acknowledgements
\begin{acknowledgements}
\noindent
The three and half years of PhD has been a wonderful adventure, and here is to present my appreciation for the people who are part of my journey.
\\
\\Firstly, I want to thank my supervisor, Christopher Lester. The whole work of my PhD will not be possible without his support and help. And, I also want to show my gratitude to my college, Peterhouse, for funding my PhD and enriching my life out of research. 
\\
\\Then, I would like to thank Takuya Nobe, Viviana Cavaliere, and Lailin Xu as the convenors for the analysis works I was involved in. Those tasks are complicated like a maze, and they provide the guidance to show where is the way to go to have the final results. I would also like to thank Ben Carlson for all my L1Calo works, because I cannot keep working on the L1Calo upgrade project without his support.  I also want to present my appreciation to John Chapman, John Hill, and Will Buttinger for all the helps on the technical problems I encountered when dealing with varied software tools.
\\
\\For my social life during PhD, I am glad to be part of the friendly Cambridge HEP group, and I want to thank everyone in the group for the sweet tea time and nice chats. Especially, I want to thank James Cowley, Alison Tully, Ben Brunt, Jonathan Rostén, Holly Pacey, and Herschel Chawdhry for the company. They have made the research work not so intense, as I can always have fun with those people. I also want to say thank you to the Cambridge Taiwanese Society, especially Cheng-Tai Lee, because this is where I can still keep the connection to where I am from even though I am thousands of miles away from home. Most importantly, I can get rid of the stress by complaining in my mother tongue with those people.
\\
\\For the last part of this list, I would like to thank my family for supporting me to be away for the pursuit of this degree. 

\end{acknowledgements}



%% ToC
\tableofcontents


%% Strictly optional!
\frontquote{%
  We are just an advanced breed of monkeys on a minor planet of a very average star. But we can understand the Universe. That makes us something very special.}%
  {Stephen Hawking}
%% I don't want a page number on the following blank page either.
\thispagestyle{empty}
