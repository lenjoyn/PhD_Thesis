\chapter{Thesis Remarks}
\chapterquote{Git-r-done}{Mater, Cars}
The Standard Model has been a successful description for the constituents of this universe giving precision predictions of how the matters interact with each other. However, a couple of puzzles still remain unsolved, which implies that the Standard Model should be corrected. Therefore, the Large Hadron Collider was built to investigate into those mysteries, and the ATLAS detector plays the role as the microscope to look into the interactions which happen in the collisions.
\\
\\New models are proposed in the attempt to solve those problems to complete the SM, and they also predict the existence of new particles most of which have the couplings to the SM bosons. The $WV\to \ell\nu qq$ final state is therefore chosen to investigate those new models including the heavy-mass Higgs boson, the heavy vector triplet, and also the RS graviton. This analysis has looked into two production modes, VBF and ggF/DY, along with two jet topologies. The analysis strategy was to employ the simulation for the SM background modelling and the fake factor method for the multijet background modelling. After the comparison between data and background estimation, no new physics is found evident with the statistical interpretation, and the exclusion limits are set on the mass of the new particles. To enhance the sensitivity of this search task, the result was combined with the other diboson and dilepton resonance final states. Unfortunately, there is still no discovery of any new particle, and the mass limits are further updated with the new limit on couplings between the HVT and the SM particles. 
\\
\\In addition to the search for new particles, another approach for the new physics is to verify the Standard Model prediction toward the interaction cross-section. This study inherited the framework from the resonance search, and it is dedicated for the vector boson scattering cross-section measurement which has the semileptonical final states ($pp\to VVjj \to \ell\nu qq$). The final result was combined with the $\ell\ell qq$ and $\nu\nu qq$, and it made the first measurement to the semileptonical final state for the VBS interaction present great agreement with the SM estimation.
\\
\\Although there is still no BSM physics discovered, the ATLAS detector has been through a fruitful Run~2 operation delivering a significant physics results. To enhance the sensitivity to new physics, the LHC will undergo the upgrade to increase both the energy and luminosity. To incorporate this upgrade, the ATLAS calorimeter hardware trigger will be implemented with new components to process the digitized LAr detector signatures. With the software simulation on this system, three physical object reconstruction algorithms are proposed showing great the same or improved performance for the upcoming Run~3 collision environment. However, there is still the great potential for more complicated algorithms to achieve better even better performance for pile-up suppression. 
\\
\\In the following decades, the ATLAS detector is expected to collect the data up to 3000~$fb^-1$ with hardware upgrades including a full silicon innter tracking system (ITK), an entirely digitized calorimeter reader, and a two-level hardware trigger (L0 and L1). This will shed the light for the underlying new physics and provide a better understanding to the Stand Model at the frontier of human knowledge and technology. 
\newpage
\pagequote{Thanks for the adventure. Now go have a new one}{Ellie, Up}