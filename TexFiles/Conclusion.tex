\chapter{Thesis Remarks}
\chapterquote{Git-r-done}{Mater, Cars}
The Standard Model has been a successful description for the constituents of this universe giving precise predictions of how the matters interact with each other. However, we saw that a few puzzles still remain unsolved. I have described how the Large Hadron Collider and the ATLAS detector were built to investigate those mysteries.
\\
\\New models were proposed in the attempt to solve those problems to complete the SM, and they also predicted the existence of new particles most of which have the couplings to the SM bosons. The $WV\to \ell\nu qq$ final state was therefore chosen to investigate those new models including the heavy-mass Higgs boson, the heavy vector triplet, and also the RS graviton. This analysis has looked into two production modes, VBF and ggF/DY, along with two jet topologies. The analysis strategy was to employ the simulation for the SM background modelling and the fake factor method for the multijet background modelling. After the comparison between data and background estimation, no evidence for new physics was found, and exclusion limits were set on the masses of the new particles. To enhance the sensitivity of this search task, the result was combined with the other diboson and dilepton resonance final states. Unfortunately, there was still no discovery of any new particle, and mass limits were further updated with the new limit on couplings between the HVT and the SM particles. 
\\
\\In addition to the search for new particles, another way of looking for new physics was to verify the Standard Model predictions toward the interaction cross-sections. This study inheriting the framework from the resonance search, and it was dedicated to the vector boson scattering cross-section measurement which has the semileptonical final states ($pp\to VVjj \to \ell\nu qq$+jj). The final result was combined with the other channels, $\ell\ell qq$ and $\nu\nu qq$, and it made the first measurement on the VBS interaction with semileptonical final states, which showed agreement with the SM estimation.
\\
\\Although no BSM physics was discovered, the ATLAS detector has been through a fruitful Run~2 operation delivering significant physics results. To enhance the sensitivity to new physics, the LHC will undergo the upgrade to increase both the energy and luminosity. To achieve better performance for physics analyses, the ATLAS will also upgrade the hardware calorimeter trigger with new components to process calorimeter signal. I have described my contributions to the software implementation for the simulation of this system including the trigger tower identification and construction. Under this framework, three preliminary physical object algorithms including my proposal of L1 $E^{miss}_{T}$ reconstruction showed the same or improved performance with the upcoming Run~3 collision environment in comparison to the Run~2 L1 objects. However, it should be noted that there is still the great potential for more complicated algorithms to achieve better performance for pile-up suppression. 
\\
\\In the following decades, the ATLAS detector is expected to collect the data up to 3000~$fb^{-1}$ with hardware upgrades including a full silicon inner tracking system (ITK), an entirely digitized calorimeter readout electronics, and a two-level hardware trigger (L0 and L1). This will shed the light for the underlying new physics and provide a better understanding to the Stand Model at the frontier of human knowledge and technology. 
\newpage
\pagequote{Thanks for the adventure. Now go have a new one}{Ellie, Up}