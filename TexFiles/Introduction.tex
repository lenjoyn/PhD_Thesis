\chapter{Introduction and Motivation}
\label{Chap:intro}
\chapterquote{We can't focus on what's going wrong, there's always a way to turn things around.}{Joy, Inside Out}
Particle physics is the subject to study the fundamental structure of the universe. It is now based on the theory called the "Standard Model" (SM). It interprets the universe as the composition of tiny particles interacting with each other by the exchange of force carriers (another type of particle).  In July 2012, the discovery of Higgs boson made by the ATLAS and CMS collaborations \cite{HIGG-2012-27,CMS-HIG-12-028} completed SM 50 years after being predicted for the existence. By now, it has been deemed as one of the most successful theories in modern physics.
\\
\\
However, there are still some conflicts between the SM and factual results. For example, in the SM, neutrinos are supposed to be massless, but the discovery of neutrino oscillation support the fact that neutrinos are massive, and the SM cannot explain it. New theories are proposed in order to resolve those conflicts, and they indicate the existence of some new particles or the deviation from SM predictions. This thesis is dedicated to the work in search for this kind of new physics.
\section{Standard Model\cite{Griffiths,Perkins}}
The SM is a quantum field theory (QFT). In the QFT, the universe is filled with different fields, and all fundamental particles (particles without further substructure) are the forms of quantized fields. They make up the matters and also mediate interactions between them, which is the foundation how this universe operates. Those fundamental particles could be classified into two types: fermions and bosons. Fermions are the matter builders, while bosons are the force carriers exchanged between particles (for both fermions and bosons). 
\\
\\{\bf Fermions}
\\
\\Fermions are quantized from fermionic fields following Dirac-Fermi statistics with half integer spin number, $\pm\frac{1}{2}$. Under the statistic characteristics, fermions exclude each other with the same quantum status in a bound state, which makes them different from bosons. \\
\\
All fermions have their antiparticles which have opposite charge and chirality. Those fermions are called "Dirac Fermions". They can be presented as Weyl spinors of four components composed of one left-handed spinor and one right handed spinor following the Dirac equation. However, neutrinos, a sub-specie of fermions, have no anti-partner with opposite chirality found\footnote{Due to being neutral, although neutrinos and anti-neutrinos were discovered, neutrinos (anti-neutrinos) only have the left-handed (right-handed) chirality.}, so they are now taken as candidates of "Majorana Fermions": they are their own antiparticle following the Majorana euqation. They could be presented as Majorana spinors in Majorana equation. 
\\
\\{\bf Dirac Equation}: $i\hbar\gamma^{\mu}\partial_{\mu}\psi-mc\psi=0$ 
\\
\\{\bf Majorana Equation}: $i\hbar\gamma^\mu\partial_\mu\psi-mc\psi_c=0$
\\
\\$\psi$ is the fermion field with charge conjugate $\psi_c$, and $\gamma^\mu$ is the gamma matrix and m is the particle mass.
\\ 
\\Fermions can then be further categorized into two types, quarks and leptons, by the interactions they participate in. Quarks are the only particles involved in the strong interaction, so they cannot exist alone, and, instead, they are always in bound state with two or more quarks. 
\\
\\Quarks have three generations and six flavours. In each generation are quarks with different charges: $\frac{-1}{3}$e and $\frac{2}{3}$e with e as the electric charged carried by electrons. The first generation are the lightest: up and down. Strange and charm are in the second generation. The third generation has bottom and top with highest mass. Quarks change their flvaours via the weak interaction, and the couplings between flavours is described by the CKM matrix which is shown in Fig.\ref{Fig:quarks}. A stronger coupling between two quarks indicates a higher possibility that the heavier quark would decay to the lighter one. 
\\    
\begin{figure}[!h]                
	\includegraphics[width=0.45\textwidth]{Chapter1/quarks.png}
	\centering
	\begin{center}
		\caption{The coupling strengths between quarks are determined by CKM matrix taken from \cite{ckm_timothy}}
		\label{Fig:quarks}            
	\end{center}
\end{figure}

Similar to quarks, leptons also have 3 generations and 6 flavours. In each generation, there is one neutral neutrino and corresponding charged particle with charge $-1$. The three generations are electrons, muons and taus with their neutrino partners. The flavour change is via the weak interactions which has one charged lepton with its partner neutrino and a W boson participating in the process. The neutrinos could also change their own flavours via neutrino oscillation described by the PMNS matrix without the company of other particles. Leptons participate in weak interaction, quantum electrodynamics(except for neutral neutrinos) and gravity.
\\
\\{\bf Interaction and Bosons}
\\
\\Under the SM, the interactions between particles are induced by gauge fields which could be quantised into gauge bosons. Different from fermions, those bosons follow Bose-Einstein statistics with their spin as integer numbers, which means more than one boson is allowed to occupy a single quantum state in a bound state. They mediate interactions between particles including themselves. 
\\
\\Although there are four fundamental forces in the universe, only three of them are in the SM, because they are quantizable: electromagnetic, weak and strong interactions. The challenge of quantizing gravity is still not achieved in the modern physics. Each interaction has a corresponding term in the SM Lagrangian.
\\
\\The electromagnetic interaction is the best known among the four interactions. It is explained by quantum electrodynamics in the SM. The interaction is induced by electromagnetic field which could be seen as the interaction between photons and charged particles, under which the electric charges are conserved as an invariance of $U(1)$ symmetry. In electromagnetic interactions, photons don't interact with neutral particles at the leading order \footnote{With a loop diagram, it can still be achieved by exchanging charged fermions between photons}. The coupling constant (a number that determines the strength of the force exerted in an interaction) in the interaction is:
\\
\begin{equation}
\alpha_{EM} = \frac{e^2}{4\pi\epsilon_0\hbar c}=\frac{1}{137.036...}
\end{equation}
\\
with $e$ as electric charge of electron, $\hbar$ as reduced Plank constant and $c$, the speed of light. Its part in the SM Lagrange could be written as:
\begin{equation}
\mathcal{L}=\bar{\psi}(i\gamma^{\mu}D_{\mu})\psi-F^{\mu\nu}F_{\mu\nu}
   \label{Eq:EM Lagrange}  
\end{equation}
with $\psi$ as the Wyle spinor of spin, $\pm \frac{1}{2}$, and $D_\mu=\partial_\mu+ieA_\mu+ieB_\mu$ representing the gauge covariant derivative with $A_\mu$ as the field induced by the particle itself and $B_\mu$ as the field from external source. In the equation, $F_{\mu\nu}$ is the electromagnetic field tensor. 
\\
\\All the left-handed particles participate in the weak interaction. It is mediated by three different bosons: the $W^+$, $W^-$ and $Z^0$ bosons. They are massive gauge bosons which obtain their mass via the electroweak symmetry breaking. The flavour change of a particle is through the weak interaction mediated by W bosons which would also involve the change of electric charge (the neutrino oscillation is an exception, as the neutrinos change their flavour without the involvement of W bosons),  while Z boson is involved in the neutral current interactions within which both the electric charges and particle flavours are conserved. In the weak interaction, a quantity, weak isospin, is conserved under $SU_L(2)$ symmetry. Its definition is similar to the spin numbers of a pair of electrons in the same orbital. For two fermions in the same generations, they could be grouped into a weak isospin doublet with $I_3=\pm\frac{1}{2}$. As right-handed fermions do not participate in weak interaction, their weak isospin is 0. The weak isospins of fermions are showed in Table.\ref{Tab:isospin}. 
\\


\begin{table}[h]
	\caption{Weak Isospin of Elementary fermions}
	\renewcommand{\arraystretch}{1.2}
	\centering
	\begin{tabular}{l c c c c r}
		\hline
		\hline
		1st Generation & $I_3$        &2nd Generation    &$I_3$        &3rd Generation   &$I_3$\\
		\hline
		$e^-$          &$-\frac{1}{2}$ &$\mu^-$           &$-\frac{1}{2}$ &$\tau$           &$-\frac{1}{2}$\\
		\hline
		$\nu_e$        &$\frac{1}{2}$  &$\nu_\mu$         &$\frac{1}{2}$  &$\nu_\tau$       &$\frac{1}{2}$\\
		\hline
		$u$            &$\frac{1}{2}$  &$c$               &$\frac{1}{2}$  &$t$              &$\frac{1}{2}$\\
		\hline
		$d$            &$-\frac{1}{2}$ &$s$               &$-\frac{1}{2}$ &$b$              &$-\frac{1}{2}$\\
		\hline

	\end{tabular}
    \label{Tab:isospin}
\end{table}
\noindent
The coupling constant for weak interaction is defined as:
\begin{equation}
\alpha_{W} = \frac{g_{W}}{4\pi\hbar c}\approx\frac{1}{29}
\end{equation}
with $g_{W}$ as the W weak charge strength. In terms of the interactions via Z boson, it is substituted by Z weak charge, $g_{Z}$. A unification between the weak and electromagnetic interactions is achieved with another new parameter called electroweak hypercharge defined as $Y_w=2 (Q-I_3)$ where $I_3$ is the isospin and $Q$ is the electric charge under $SU_L(2)\times U(1)$ symmetry in the scale of high energy.  In the SM, the symmetry is spontaneously broken by the Higgs field to give particles mass. It will be discussed in the next section.
\\
\\Only quarks are involved in the strong interaction which is described by quantum chromodynamics (QCD). The conserved quantity in the interaction is the colour charge, with gluons as the force carrier boson under $SU(3)$ symmetry. There are three different colours: red, blue and green along with their anti-colour partners. Similar to the colour principle of light, the colour would be neutral (white) when the three colours are mixed together or with their anti-colour, and it is the condition for a stable state in QCD. However, each quark is only allowed to carry one colour, and this is an unstable state. It needs to be bound with another quark(s) to stabilize the system, which is called colour confinement. In QCD, gluons have 8 types with different colour combinations:
\begin{equation}
         (r\bar{b}+b\bar{r})/\sqrt{2}, \quad -i(r\bar{b}-b\bar{r})/\sqrt{2}  \\
         (r\bar{g}+g\bar{r})/\sqrt{2}, \quad -i(r\bar{g}-g\bar{r}/\sqrt{2})  \\
          (b\bar{g}+g\bar{b})/\sqrt{2},\quad -i(b\bar{g}-g\bar{b}/\sqrt{2})   \\
          (r\bar{r}-b\bar{b})/\sqrt{2},\quad  (r\bar{r}+b\bar{b}-2g\bar{g})/\sqrt{6}
\end{equation}
with $r$, red charge, $b$, blue charge, and $g$, green charge.
\\
\\Its part of the SM Lagrange is shown as:
\begin{equation}
\mathcal{L}_{QCD}=\bar{\psi}(i(\gamma^\mu D_\mu)_{ij}-m\delta_{ij})\psi_j-\frac{1}{4}G^a_{\mu\nu}G_a^{\mu\nu} 
\end{equation}
with
\begin{equation}
G^a_{\mu\nu}=\partial_\mu\mathcal{A}^a_\mu-\partial_\nu\mathcal{A}^a_\nu+gf^{abc}\mathcal{A}^b_\mu\mathcal{A}^c_\nu 
\end{equation}
with $\psi_i$, the quark field in $SU(3)$ representation indexes of i,j, ..., $G^a_{\mu\nu}$, the gluon field also in $SU(3)$ representation indexed of a, b... from 1 to 8. $f^{abc}$ is the structure constant, $\mathcal{A}_\mu$ is the spin 1 gluon field and $g = \sqrt{4\pi\alpha_s}$ is the QCD coupling strength with $\alpha_s$ as the fine structure constant. It should be noted that the coupling strength is not a constant in QCD but dependent on colour-charged particle energy due to the effect of the gluon self-interaction. It leads to the result that the interaction between colour-charged particles are strong, when they carry lower energy. In the other opposite way, the quarks would behave like free particles when they have higher energy, as the interaction strength is weak in this scenario, which is called ``asymptotic freedom''.
\\
\\All the elementary particles with their basic properties are shown in Fig. \ref{Fig:elementary_particle}. The 3 interactions with their conserved quantities makes the SM a gauge quantum field theory containing the internal symmetries of the unitary product group, $U(1) \times SU(2)_L \times SU(3)$.

\begin{figure}[!h]                
	\includegraphics[width=0.9\textwidth]{Chapter1/elementary_particle.png}
	\centering
	\begin{center}
		\caption{Elementary particles and properties taken from \cite{SMParticle}}
		\label{Fig:elementary_particle}            
	\end{center}
\end{figure}

\section{Electroweak Symmetry Breaking}
One particle in Fig. \ref{Fig:elementary_particle} is not mentioned yet: Higgs boson, the last discovered fundamental particle in the SM. It arises from quantised Higgs field which was proposed by three groups in early 1960s:  Robert Brout and Francois Englert\cite{Englert}, Peter Higgs\cite{Higgs} as well as Gerald Guralnik, C. R. Hagen, and Tom Kibble\cite{Hagen}. It induces spontaneous electroweak symmetry breaking via the ''Brout-Englert-Higgs mechanism``.  The Higgs boson discovery was announced on 4th July 2012 and confirmed on 14 March 2013 with spin 0 and $+$ parity by the ATLAS and CMS collaborations.
\\
\\The Higgs field is defined as the scalar gauge field in a complex scalar $SU(2)_L$ doublet.
\begin{equation}
 \Phi= \left(  \begin{array}{ c } \phi^+\\  \phi^0 \end{array} \right) 
\end{equation}
with both $\phi^0$ and $\phi^+$ as arbitrary imaginary numbers representing the neutral and charged components. The potential for this field is then given as:
\begin{equation}
 V(\Phi)=\mu^2|\Phi^\dagger\Phi|+\lambda(|\Phi^\dagger\Phi|)^2
\label{Eq:sm_higgs_potential}
\end{equation}
Here, $\mu$ and $\lambda$ are arbitrary constants, and some choices of them could make the potential minimum at $\Phi = 0$. For this case, the shape of potential would be seen in Fig.~\ref{Fig:V} (this is a simplified plot, and the real one should be in 4 dimensions). In this potential, the symmetry is not broken with the minimal value at $\Phi = 0$.
\\ 
\begin{figure}[!h]                
	\includegraphics[width=0.4\textwidth]{Chapter1/V.pdf}
	\centering
	\begin{center}
		\caption{Scalar potential with $\mu^2 > 0$}
		\label{Fig:V}            
	\end{center}
\end{figure}
\\In an alternative scenario for $\mu^2 < 0$, the potential shape becomes Fig. \ref{Fig:higgs}. The minimal expected value of the potential is not at 0 but at:
\begin{equation}
\label{Eq:min}
\langle\Phi\rangle=\sqrt{-\frac{\mu^2}{2\lambda}}\left(  \begin{array}{ c } 0 \\ 1\end{array} \right) \equiv\frac{\nu}{\sqrt{2}}\left(  \begin{array}{ c } 0 \\ 1\end{array} \right)
\end{equation}
with $\nu$ as the ``vacuum expected value'' (VEV). It could be noted that the term of $\phi^{}+$ is shifted to zero with a rotation on the phase space of $(\phi^{+},\phi^0)$. To maintain a stable state, particles are only allowed to stay in the lowest potential, the valley part. This makes the degree of freedom of the particles decrease from four to one and breaks the $SU_L(2)\times U(1)$ symmetry with isospin and hypercharge to $U(1)$ symmetry with electric charge.  
\begin{figure}[!h]                
	\includegraphics[width=0.4\textwidth]{Chapter1/higgs.pdf}
	\centering
	\begin{center}
		\caption{Scalar potential with $\mu^2 < 0$}
		\label{Fig:higgs}            
	\end{center}
\end{figure}
In high energy regime above the valley (excited state), electromagnetic and weak interaction are mixed together to form three $SU_L(2)$ gauge bosons, $W^i_\mu$ with $\mu =1,2,3$ and one $U(1)$ gauge boson, $B_\mu$. They are not SM particles, but they could be taken as the excited form of SM gauge bosons. The Lagrangian for the interaction between them and Higgs field is:
\begin{equation}
 \mathcal{L}=(D^\mu\Phi)^\dagger(D_\mu\Phi) - V(\Phi) 
\end{equation}
with
\begin{equation}
 D_\mu = \partial_\mu+i\frac{g}{2}\tau\cdot W_\mu+i\frac{g'}{2}B_\mu Y
 \label{Eq:electroweak symmetry Lagrange}  
\end{equation}
$g$ and $g'$ are the coupling constants between the fields respectively, $\tau$ is the Pauli matrix and Y is the hypercharge.
\\
\\A unitary gauge transformation on the Higgs field can remove Goldstone bosons\footnote{Unitary gauge transformation is to select the fixed gauge which sets the Goldstone boson terms into 0} after the symmetry breaking. The Higgs field is thus shifted with the new gauge as:
\begin{equation}
\langle\Phi\rangle=\frac{\nu+h}{\sqrt{2}}\left(  \begin{array}{ c } 0 \\ 1\end{array} \right)
\end{equation}
with h, the physical Higgs sector, as a real number.
\\
\\After inserting the new Higgs field into and rearranging SM Lagrangian, the SM gauge bosons could be shown as:  
\begin{equation}
W^{\pm}_\mu=\frac{1}{\sqrt{2}}(W^{1}_\mu\mp iW^{2}_\mu) 
\end{equation}
\begin{equation}
Z^\mu=\frac{-g'B_\mu+gW^3_\mu}{\sqrt{g^2+g'^2}} 
\end{equation}
\begin{equation}
A^\mu=\frac{gB_\mu+g'W^3_\mu}{\sqrt{g'^2+g^2}} 
\end{equation}
with particle masses:
\begin{equation}
M^2_W=\frac{1}{4}g^2\nu^2 
\end{equation}
\begin{equation}
M^2_Z =\frac{1}{4}(g^2+g'^2)\nu^2 
\end{equation}
\begin{equation}
M_A=0 
\end{equation}
From the expression, it turns out that $Z$ boson and photon are both the mix of $B$ and $W^3$ bosons with different phases which could be shown as:
\begin{equation}
 \left[ \begin{array}{c}  A \\ Z \end{array} \right]=\left[ \begin{array}{l r} \cos\theta_W &  \sin\theta_W \\ -\sin\theta_W & \cos\theta_W \end{array} \right]\left[ \begin{array}{c}  B\\ W^3 \end{array} \right] 
  \label{Eq:AZ phase}  
\end{equation}
With $\cos{\theta_W}=\frac{g}{\sqrt{g^2+g'^2}}$ and $\sin{\theta_W}=\frac{g'}{\sqrt{g^2+g'^2}}$. Here, $\theta_W$ is called the weak mixing angle or Weinberg angle. By this, the electroweak parameter, $\rho$, is defined:
\begin{equation}
\label{Eq:rho}
\rho = \frac{m_W}{m_Z\cos{\theta_W}} 
\end{equation}
\\with the comparison between Eq. \ref{Eq:EM Lagrange} and Eq. \ref{Eq:electroweak symmetry Lagrange} with Eq. \ref{Eq:AZ phase}, the electric charge could be defined as:
\begin{equation}
e=g\sin{\theta_W}=g'\cos{\theta_W}
\end{equation}
\\This relation gives the access to a precision measurement of $\rho$, which is now given ~1.0008, a litte deviation from expectation of 1 in the SM because of the loop diagram correction.
\\
\\In terms of degrees of freedom, before symmetry breaking, it comes with four degrees from Higgs complex scalar doublet, six degrees from $SU(2)_L$ gauge fields, $W_i$, and two degrees from $U(1)_Y$ gauge field, $B$, which makes 12 degrees in total for all the massless fields. After symmetry breaking, the number of degrees of freedom does not reduce with nine degrees from three massive vector boson, $Z$ and $W_{\pm}$, two degrees from massless photon, A, and one degree from physical real scalar field, $h$.
\\
\\Not only granting mass to bosons, the interaction between fermions and Higgs boson is also part of the Brout-Englert-Higgs Mechanism. The left-handed fermionic field is defined as a doublet:
\begin{equation}
 Q_L=\left(  \begin{array}{ c } u_L\\  d_L \end{array} \right)
\end{equation}
For right-handed ferions, the representation would be in a singlet, $u_R$ and $d_{R}$, due to the fact that they have different transformation under the $SU(2)\times U(1)$ gauge symmetry.      
\\
\\Their interaction with Higgs field are through Yukawa couplings\footnote{Yukawa coupling means the couplings betweent fermionic and bosonic fields}
\begin{equation}
\mathcal{L} = -\lambda\bar{Q_L}\Phi d_R + h.c. 
\end{equation}
with $\lambda$ as the coupling constant. The Lagrangian can lead to the fermionic mass as:
\begin{equation}
 m_d=\frac{\lambda \nu}{\sqrt{2}}
\end{equation}
This mechanism would change the chirality of a fermion, when it is giving the mass. However, no right-handed neutrino and left-handed anti-neutrino were measured, which leaves it as one of the unsolved problem in SM. (More details are given in next section.)
\section{Unsolved Problems in SM}
With SM, we have understood most behaviours of the fundamental particles. However, it still failed explaining some experimental results. The following is part of them the work in the thesis is trying to answer.
\\
\\{\bf Higgs Mass Naturalness\cite{WILLIAMS201582}}
\\
\\In quantum field theory, all the experimental observables could be presented as:
\begin{equation}
O=a_1+a_2+a_3+...
\end{equation}
where O corresponds to the physical observables like the invariant mass of particles, and $a_n's$ are the independent contributions to the observables. For naturalness of the observable, it is expected that $a_n\leq O$. For any case that $a_n>>0$, the further fine-tuning needs to be introduced for proper correction on theory, and it also indicates the defect in the theory. \\
\\The form for the observable of Higgs mass is:
\begin{equation}
m_h^2=2\mu^2+\delta m_h^2
\end{equation}
where $\delta m_h^2$ for the contribution from coupling to top quark is:
\begin{equation}
\delta m_h^2 \simeq \frac{3}{4\pi^2}\left(\lambda^2_t+\frac{g^2}{4}+\frac{g^2}{8\cos^2{\theta_w}}+\lambda\right)\Lambda
\end{equation}
where $\lambda_t$ is the top-quark Yukawa coupling, and $\Lambda$ is the energy cut-off to divergent loop integrals. With the observed Higgs boson mass at 125~$GeV$, $\Lambda$ is estimated to be around 1~$TeV$, and that is also roughly the limit to keep the naturalness of this observable. 
\\
\\However, many models beyond the SM predict the existence of particles at the TeV scale, which means the naturalness would be broken in the scenario. For this reason, a correction to Brout-Englert-Higgs Mechanism is needed, or there is possibly a heavier Higgs boson to complete the theory.  
\\
\\{\bf The Hierarchy Problem and Quantum Gravity\cite{BenA}}
\\
\\The hierarchy problem is defined in two ways: the unreasonable discrepancy between theoretical prediction and experimental result, or two comparable parameters. Higgs mass is one instance for the first definition. For the second one, it is generally referred to the gap between coupling strengths of weak interaction and gravity in the order of $10^{16}$.
\\
\\When a hierarchy problem occurs, the ``so-called'' fine-tuning is introduced to correct the discrepancy between two parameters. However, the fine-tunning could only be performed with enough understanding on the quantum effect of related parameters, and quantum gravity is still an unsolved problem. In the case, no solution is available.
\\
\\{\bf Neutrino Mass}
\\
\\Brout-Englert-Higgs Mechanism is the process to make particles massive within which the chirality of fermions would be changed. This implies that massive fermions of right-handed and left-handed chirality shall both exist, but no evidence is found for right handed neutrinos (or left-handed anti-neutrinos). Therefore, they are supposed be massless with SM. However, with the measurement of neutrino oscillation\cite{SuperK} induced by the difference of neutrino mass and flavour eigenstates, they are practically massive particles. The conflict between SM and experiment still remains unsolved.
\section{Thesis Overview}
To solve the problems in SM, analyses are performed in two ways, direct and indirect searches which are corresponding to two different signatures in physics: new particles or new interactions. The thesis will present how the experiment is set up to see the signatures of new physics in Chapter \ref{chap:exp}, and the following three chapters are dedicated to show the analyses of these two types of signatures with 2015+2016 data corresponding to the integrated luminosity of $36.1~fb^{-1}$ for which I made the contributions to the multijet background estimation, study on trigger performance, data background comparison, analysis framework development, and the statistical interpretation. The last chapter is for the simulation of the upgrade of the LHC and ATLAS detector which will start to operate in 2021 for which I made the contribution to the construction of the simulation framework and also the study for the preliminary missing transverse energy ($E^{miss}_{T}$, the definition will be shown later) trigger.

